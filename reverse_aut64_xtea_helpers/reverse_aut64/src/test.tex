\documentclass{template}

\project{Graduation Project II}
\title{Put your project title here}
\author{Name1 Surname1\\Name2 Surname2}
\faculty{Engineering}
\department{Mechanical Engineering}
\abst{Put your abstract here.}

% Remove these lines if do not have nomenclature.
\nomenclature{$\delta$}{Planck constant}
\nomenclature{$c$}{Speed of light}

\begin{document}

\chapter{Introduction}\label{ch:in}

If you want to write an equation, you can use

\begin{equation}\label{eq:A}
A = \rho + B\eta^6
\end{equation}

To cross-reference the equation above, use the label of the equation such as \Cref{eq:A}.

For multi-line equation use

\begin{equation}
\begin{aligned}
A &= \rho + B\eta^6\\
  &= \int_{f=0}^{f=6} \sinh(x)
\end{aligned}
\end{equation}

For inline equation use $A = \rho + B\eta^6$.

To use units with numbers do

\begin{equation}
\begin{aligned}
A &= \SI{5}{\kg\per\s}\\
A &= \SI{5e-3}{\joule\per\kelvin\s}\\
A &= \num{5e-3}\\
\end{aligned}
\end{equation}


\section{Problem statement}

The file \texttt{ref.bib} contains your references. You can use google scholar to copy BibTeX output to \texttt{ref.bib}. Here is how you cite \cite{gulawani2006cfd}.

\section{Aim and objectives}

Here is how you include a figure. The figure will be placed to some appropriate location. This is most of the time recommended. If, for some reason, you want to put figure right under this paragraph, use [H] option in \textbackslash begin\{figure\}[H].

\begin{figure}
\caption{Put your caption here.}
\label{fig:logo}
\includegraphics[scale=0.2]{logo}
\end{figure}

Now I am going to cross-reference the figure by using its label \Cref{fig:logo}.

\section{Scope and limitations}

This is how you make a table.

\begin{table}
\caption{Put table caption here.}
\label{tab:city}
\entries{
person  & singEnglish & singGaeilge & pluralEnglish & pluralGaeilge\\
    1st & at me       & agam        & at us         & againn\\
    2st & agat        & at you      & agaibh        & other \\ 
    3rd & at him      & aige        & at them       & acu\\
        & at her      & aici        &               &\\
}
\end{table}

Let's cross-reference to \Cref{tab:city}.

\chapter{Literature review}\label{ch:lr}

You can cross-reference to a chapter or section by using its label. Let's cross-reference to Literature Review chapter such as \Cref{ch:lr}.

\chapter{Methodology}\label{ch:me}
\chapter{Results}\label{ch:re}
\chapter{Conclusion}\label{ch:co}

% You do not need to remove this even if you do not have appendices.
\appendix % Anything you add will belong to appendix from this point on.

\end{document}
